% main.tex - Documento principal del Proyecto Modular 3
\documentclass[12pt,letterpaper]{article}

% Paquetes esenciales
\usepackage[utf8]{inputenc}
\usepackage[spanish]{babel}
\usepackage{amsmath,amssymb}
\usepackage{graphicx}
\usepackage{hyperref}
\usepackage{cite}
\usepackage{geometry}
\geometry{margin=2.5cm}

% Información del documento
\title{Simulación Monte Carlo de Aceleradores Lineales con Denoising mediante Redes Neuronales Profundas}
\author{
    Proyecto Modular 3 \\
    CUCEI - Universidad de Guadalajara \\
    Licenciatura en Física
}
\date{Enero 2026}

\begin{document}

\maketitle
\tableofcontents
\newpage

\section{Introducción}

La radioterapia externa mediante aceleradores lineales (linacs) es una de las modalidades más utilizadas para el tratamiento del cáncer. La planificación precisa de tratamientos requiere cálculos de dosis exactos, tradicionalmente realizados mediante simulaciones Monte Carlo que modelan el transporte de radiación de manera estocástica.

Este proyecto implementa:
\begin{itemize}
    \item Simulación completa de un linac de 6 MV usando GATE 10 (Geant4 11.3.2)
    \item Denoising de distribuciones de dosis mediante redes neuronales convolucionales 3D (MCDNet)
    \item Validación mediante análisis gamma index con criterios clínicos (3\%/3mm, 2\%/2mm)
\end{itemize}

\subsection{Objetivos}

\textbf{Objetivo General:} Desarrollar un sistema computacional que combine simulaciones Monte Carlo de aceleradores lineales con técnicas de aprendizaje profundo para acelerar cálculos de dosis manteniendo precisión clínica.

\textbf{Objetivos Específicos:}
\begin{enumerate}
    \item Modelar geometría completa de linac 6 MV con GATE 10
    \item Implementar generación y reutilización de espacios de fase
    \item Entrenar red neuronal MCDNet para denoising de dosis
    \item Validar resultados con criterios gamma $\geq 95\%$ pass rate
\end{enumerate}

\section{Marco Teórico}

\subsection{Simulación Monte Carlo}

El método Monte Carlo simula el transporte de partículas mediante muestreo aleatorio de procesos físicos. Para fotones y electrones en radioterapia, los procesos principales son:

\begin{itemize}
    \item \textbf{Fotones:} Efecto fotoeléctrico, dispersión Compton, producción de pares
    \item \textbf{Electrones:} Ionización, bremsstrahlung, dispersión múltiple
\end{itemize}

GATE 10 utiliza Geant4 11.3.2 con physics list \texttt{QGSP\_BIC\_EMZ}, optimizada para aplicaciones médicas con modelos electromagnéticos de alta precisión.

\subsection{Arquitectura MCDNet}

MCDNet es una red neuronal convolucional 3D diseñada para denoising de dosis Monte Carlo. Características:

\begin{equation}
    \text{MCDNet}: \mathbb{R}^{D \times H \times W} \rightarrow \mathbb{R}^{D \times H \times W}
\end{equation}

\begin{itemize}
    \item 10 capas convolucionales 3D sin downsampling
    \item Conexiones residuales cada 3 capas
    \item Batch normalization + ReLU
    \item Aprendizaje residual: $\hat{D}_{\text{clean}} = D_{\text{noisy}} + f_{\theta}(D_{\text{noisy}})$
\end{itemize}

\subsection{Análisis Gamma Index}

El índice gamma evalúa concordancia entre distribuciones de dosis:

\begin{equation}
    \gamma(r_r, r_e) = \min_{r_e} \sqrt{\frac{\Delta d^2(r_r, r_e)}{\Delta D_M^2} + \frac{\delta^2(r_r, r_e)}{\Delta d_M^2}}
\end{equation}

Donde:
\begin{itemize}
    \item $\Delta d$: diferencia de dosis
    \item $\delta$: distancia espacial
    \item $\Delta D_M$: tolerancia de dosis (3\% o 2\%)
    \item $\Delta d_M$: tolerancia de distancia (3mm o 2mm)
\end{itemize}

Criterio de aceptación: $\gamma \leq 1.0$ en $\geq 95\%$ de puntos evaluados.

\section{Metodología}

\subsection{Simulación del Linac}

Geometría del cabezal:
\begin{enumerate}
    \item \textbf{Blanco de Tungsteno:} Diámetro 5mm, grosor 3mm
    \item \textbf{Colimador Primario:} Tungsteno, apertura cónica
    \item \textbf{Filtro Aplanador:} Hierro, grosor 5mm
    \item \textbf{Cámaras de Ionización:} Monitorización del haz
    \item \textbf{Colimador Secundario (Jaws):} Campo $10 \times 10$ cm$^2$
\end{enumerate}

Parámetros del haz de electrones:
\begin{itemize}
    \item Energía media: 5.8 MeV
    \item FWHM energético: 3\%
    \item Spot size: $\sigma = 3$ mm (gaussiano)
\end{itemize}

\subsection{Entrenamiento de MCDNet}

\textbf{Dataset:}
\begin{itemize}
    \item Pares (dosis baja estadística, dosis alta estadística)
    \item Baja: $10^7$ historias, Alta: $10^9$ historias
    \item Fantomas: agua, hueso, pulmón
    \item Campos: $5 \times 5$, $10 \times 10$, $15 \times 15$ cm$^2$
\end{itemize}

\textbf{Entrenamiento:}
\begin{itemize}
    \item Función de pérdida: MSE (L2)
    \item Optimizador: Adam, $\alpha = 10^{-4}$
    \item Batch size: 4
    \item Épocas: 100
    \item Data augmentation: flip, rotación 90°
\end{itemize}

\subsection{Validación}

\textbf{Métricas cuantitativas:}
\begin{itemize}
    \item Gamma Index (3\%/3mm, 2\%/2mm)
    \item PSNR, SSIM
    \item Diferencias de dosis: mean, std, max
\end{itemize}

\textbf{Métricas dosimétricas:}
\begin{itemize}
    \item PDD (Percentage Depth Dose)
    \item Perfiles transversales
    \item Penumbra
    \item Índices de homogeneidad
\end{itemize}

\section{Resultados}

% Aquí se incluirán los resultados experimentales

\section{Conclusiones}

% Conclusiones del proyecto

\section{Referencias}

\begin{thebibliography}{9}

\bibitem{sarrut2021}
Sarrut, D. et al. (2021). 
\textit{Advanced Monte Carlo simulations of emission tomography imaging systems with GATE}. 
Physics in Medicine \& Biology, 66(10).

\bibitem{mcdnet2019}
Xiang, L. et al. (2019).
\textit{Ultra-fast Monte Carlo dose calculation using MCDNet, a deep learning network}.
Medical Physics, 46(11).

\bibitem{low2003}
Low, D. A. et al. (2003).
\textit{Gamma dose distribution evaluation tool}.
Journal of Physics: Conference Series, 250.

\end{thebibliography}

\end{document}
